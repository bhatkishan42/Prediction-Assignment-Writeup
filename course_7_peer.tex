\documentclass[]{article}
\usepackage{lmodern}
\usepackage{amssymb,amsmath}
\usepackage{ifxetex,ifluatex}
\usepackage{fixltx2e} % provides \textsubscript
\ifnum 0\ifxetex 1\fi\ifluatex 1\fi=0 % if pdftex
  \usepackage[T1]{fontenc}
  \usepackage[utf8]{inputenc}
\else % if luatex or xelatex
  \ifxetex
    \usepackage{mathspec}
  \else
    \usepackage{fontspec}
  \fi
  \defaultfontfeatures{Ligatures=TeX,Scale=MatchLowercase}
\fi
% use upquote if available, for straight quotes in verbatim environments
\IfFileExists{upquote.sty}{\usepackage{upquote}}{}
% use microtype if available
\IfFileExists{microtype.sty}{%
\usepackage[]{microtype}
\UseMicrotypeSet[protrusion]{basicmath} % disable protrusion for tt fonts
}{}
\PassOptionsToPackage{hyphens}{url} % url is loaded by hyperref
\usepackage[unicode=true]{hyperref}
\hypersetup{
            pdftitle={course\_7\_peer},
            pdfauthor={kishan bhat},
            pdfborder={0 0 0},
            breaklinks=true}
\urlstyle{same}  % don't use monospace font for urls
\usepackage[margin=1in]{geometry}
\usepackage{graphicx,grffile}
\makeatletter
\def\maxwidth{\ifdim\Gin@nat@width>\linewidth\linewidth\else\Gin@nat@width\fi}
\def\maxheight{\ifdim\Gin@nat@height>\textheight\textheight\else\Gin@nat@height\fi}
\makeatother
% Scale images if necessary, so that they will not overflow the page
% margins by default, and it is still possible to overwrite the defaults
% using explicit options in \includegraphics[width, height, ...]{}
\setkeys{Gin}{width=\maxwidth,height=\maxheight,keepaspectratio}
\IfFileExists{parskip.sty}{%
\usepackage{parskip}
}{% else
\setlength{\parindent}{0pt}
\setlength{\parskip}{6pt plus 2pt minus 1pt}
}
\setlength{\emergencystretch}{3em}  % prevent overfull lines
\providecommand{\tightlist}{%
  \setlength{\itemsep}{0pt}\setlength{\parskip}{0pt}}
\setcounter{secnumdepth}{0}
% Redefines (sub)paragraphs to behave more like sections
\ifx\paragraph\undefined\else
\let\oldparagraph\paragraph
\renewcommand{\paragraph}[1]{\oldparagraph{#1}\mbox{}}
\fi
\ifx\subparagraph\undefined\else
\let\oldsubparagraph\subparagraph
\renewcommand{\subparagraph}[1]{\oldsubparagraph{#1}\mbox{}}
\fi

% set default figure placement to htbp
\makeatletter
\def\fps@figure{htbp}
\makeatother


\title{course\_7\_peer}
\author{kishan bhat}
\date{August 29, 2020}

\begin{document}
\maketitle

\subsection{\texorpdfstring{\textbf{Introduction}}{Introduction}}\label{introduction}

\emph{You work for Motor Trend, a magazine about the automobile
industry. Looking at a data set of a collection of cars, they are
interested in exploring the relationship between a set of variables and
miles per gallon (MPG) (outcome). They are particularly interested in
the following two questions:}

.\emph{Is an automatic or manual transmission better for MPG}
.\emph{Quantify the MPG difference between automatic and manual
transmissions}

\subsection{Loading Data}\label{loading-data}

\begin{verbatim}
##                    mpg cyl disp  hp drat    wt  qsec vs am gear carb
## Mazda RX4         21.0   6  160 110 3.90 2.620 16.46  0  1    4    4
## Mazda RX4 Wag     21.0   6  160 110 3.90 2.875 17.02  0  1    4    4
## Datsun 710        22.8   4  108  93 3.85 2.320 18.61  1  1    4    1
## Hornet 4 Drive    21.4   6  258 110 3.08 3.215 19.44  1  0    3    1
## Hornet Sportabout 18.7   8  360 175 3.15 3.440 17.02  0  0    3    2
## Valiant           18.1   6  225 105 2.76 3.460 20.22  1  0    3    1
\end{verbatim}

\begin{verbatim}
##       mpg             cyl             disp             hp       
##  Min.   :10.40   Min.   :4.000   Min.   : 71.1   Min.   : 52.0  
##  1st Qu.:15.43   1st Qu.:4.000   1st Qu.:120.8   1st Qu.: 96.5  
##  Median :19.20   Median :6.000   Median :196.3   Median :123.0  
##  Mean   :20.09   Mean   :6.188   Mean   :230.7   Mean   :146.7  
##  3rd Qu.:22.80   3rd Qu.:8.000   3rd Qu.:326.0   3rd Qu.:180.0  
##  Max.   :33.90   Max.   :8.000   Max.   :472.0   Max.   :335.0  
##       drat             wt             qsec             vs        
##  Min.   :2.760   Min.   :1.513   Min.   :14.50   Min.   :0.0000  
##  1st Qu.:3.080   1st Qu.:2.581   1st Qu.:16.89   1st Qu.:0.0000  
##  Median :3.695   Median :3.325   Median :17.71   Median :0.0000  
##  Mean   :3.597   Mean   :3.217   Mean   :17.85   Mean   :0.4375  
##  3rd Qu.:3.920   3rd Qu.:3.610   3rd Qu.:18.90   3rd Qu.:1.0000  
##  Max.   :4.930   Max.   :5.424   Max.   :22.90   Max.   :1.0000  
##        am              gear            carb      
##  Min.   :0.0000   Min.   :3.000   Min.   :1.000  
##  1st Qu.:0.0000   1st Qu.:3.000   1st Qu.:2.000  
##  Median :0.0000   Median :4.000   Median :2.000  
##  Mean   :0.4062   Mean   :3.688   Mean   :2.812  
##  3rd Qu.:1.0000   3rd Qu.:4.000   3rd Qu.:4.000  
##  Max.   :1.0000   Max.   :5.000   Max.   :8.000
\end{verbatim}

\subsection{Analysis}\label{analysis}

Finding relation ship and assigning details for auto/man

\begin{verbatim}
## 
##  Welch Two Sample t-test
## 
## data:  mtcars$mpg by mtcars$am
## t = -3.7671, df = 18.332, p-value = 0.001374
## alternative hypothesis: true difference in means is not equal to 0
## 95 percent confidence interval:
##  -11.280194  -3.209684
## sample estimates:
## mean in group auto mean in group manu 
##           17.14737           24.39231
\end{verbatim}

\subsection{Difference between automatic and manual
transmissions}\label{difference-between-automatic-and-manual-transmissions}

\begin{itemize}
\tightlist
\item
  multivariate linear regression.
\item
  stepwise regression.
\end{itemize}

\begin{verbatim}
## 
## Call:
## lm(formula = mpg ~ wt + qsec + am, data = mtcars)
## 
## Residuals:
##     Min      1Q  Median      3Q     Max 
## -3.4811 -1.5555 -0.7257  1.4110  4.6610 
## 
## Coefficients:
##             Estimate Std. Error t value Pr(>|t|)    
## (Intercept)   9.6178     6.9596   1.382 0.177915    
## wt           -3.9165     0.7112  -5.507 6.95e-06 ***
## qsec          1.2259     0.2887   4.247 0.000216 ***
## ammanu        2.9358     1.4109   2.081 0.046716 *  
## ---
## Signif. codes:  0 '***' 0.001 '**' 0.01 '*' 0.05 '.' 0.1 ' ' 1
## 
## Residual standard error: 2.459 on 28 degrees of freedom
## Multiple R-squared:  0.8497, Adjusted R-squared:  0.8336 
## F-statistic: 52.75 on 3 and 28 DF,  p-value: 1.21e-11
\end{verbatim}

\begin{verbatim}
## 
## Call:
## lm(formula = mpg ~ factor(am):wt + factor(am):qsec, data = mtcars)
## 
## Residuals:
##     Min      1Q  Median      3Q     Max 
## -3.9361 -1.4017 -0.1551  1.2695  3.8862 
## 
## Coefficients:
##                     Estimate Std. Error t value Pr(>|t|)    
## (Intercept)          13.9692     5.7756   2.419  0.02259 *  
## factor(am)auto:wt    -3.1759     0.6362  -4.992 3.11e-05 ***
## factor(am)manu:wt    -6.0992     0.9685  -6.297 9.70e-07 ***
## factor(am)auto:qsec   0.8338     0.2602   3.205  0.00346 ** 
## factor(am)manu:qsec   1.4464     0.2692   5.373 1.12e-05 ***
## ---
## Signif. codes:  0 '***' 0.001 '**' 0.01 '*' 0.05 '.' 0.1 ' ' 1
## 
## Residual standard error: 2.097 on 27 degrees of freedom
## Multiple R-squared:  0.8946, Adjusted R-squared:  0.879 
## F-statistic: 57.28 on 4 and 27 DF,  p-value: 8.424e-13
\end{verbatim}

The results suggests that the best model includes cyl6, cyl8, hp, wt,
and amManual variables. About 86.59\% of the variance is explained by
this model. Cylinders change negatively with mpg (-3.03miles and
-2.16miles for cyl6 and cyl8 respectively), so do with horsepower
(-0.03miles), and weight (-2.5miles for every 1,000lb). On the other
hand, manual transmission is 1.81mpg better than automatic transmission.

Residual plots seems to be randomly scattered, and some transformation
may be needed for linearity

\subsection{Conclusion}\label{conclusion}

On average, manual transmission is better than automatic transmission by
1.81mpg. However, transmission type is not the only factor accounting
for MPG, cylinders, horsepower, and weitght are the important factors in
affecting the MPG.

\subsection{APPX 1}\label{appx-1}

\includegraphics{course_7_peer_files/figure-latex/unnamed-chunk-4-1.pdf}

\subsection{APPX 2}\label{appx-2}

\includegraphics{course_7_peer_files/figure-latex/unnamed-chunk-5-1.pdf}

\subsection{APPX 3}\label{appx-3}

\includegraphics{course_7_peer_files/figure-latex/unnamed-chunk-6-1.pdf}
\includegraphics{course_7_peer_files/figure-latex/unnamed-chunk-6-2.pdf}
\includegraphics{course_7_peer_files/figure-latex/unnamed-chunk-6-3.pdf}
\includegraphics{course_7_peer_files/figure-latex/unnamed-chunk-6-4.pdf}

Note that the \texttt{echo\ =\ FALSE} parameter was added to the code
chunk to prevent printing of the R code that generated the plot.

\end{document}
